\documentclass[a4paper]{article}
\usepackage[utf8]{inputenc}
\usepackage[T2A]{fontenc}
\usepackage{dsfont}
\setlength{\textheight}{25cm}
\setlength{\textwidth}{18cm}
\setlength{\topmargin}{-25mm}
\setlength{\hoffset}{-25mm}
\def\zn{,\kern-0.09em,}

\begin{document}
\thispagestyle{empty}

\begin{flushleft}
Математички факултет\\
Универзитета у Београду
\end{flushleft}

\bigskip%

\begin{center}
\textbf{МОЛБА\\
ЗА ОДОБРАВАЊЕ ТЕМЕ МАСТЕР РАДА
}\end{center}

\bigskip%

\begin{flushleft}
Молим да ми се одобри израда мастер рада под насловом:
\end{flushleft}

\begin{minipage}{16.5cm}
%%%%%%%%%%%%%%%%%%%%%%%%%%%%%%%%%%%%%%%%%%%%%%%%%%%%%%%%%%%%%%%%%%%%%%%%%%%%%%%
% U donji red upisati naziv master rada umesto teksta: >>Назив мастер рада<<  %
%%%%%%%%%%%%%%%%%%%%%%%%%%%%%%%%%%%%%%%%%%%%%%%%%%%%%%%%%%%%%%%%%%%%%%%%%%%%%%%
\textbf{\textit{\znФормализација интуиционистичке теорије типова као увод у хомотопну теорију типова''}}
\end{minipage}\\
\rule[4mm]{17.5cm}{.05mm}
\begin{flushleft}
\framebox{%
\begin{minipage}[t][10cm]{17cm}
%%%%%%%%%%%%%%%%%%%%%%%%%%%%%%%%%%%%%%%%%%%%%%%%%%%%%%%%%%%%%%%%%%%%%%%%%%%%%%%
% 	-- unutrasnjost pravougaonika --    	  								  %
%%%%%%%%%%%%%%%%%%%%%%%%%%%%%%%%%%%%%%%%%%%%%%%%%%%%%%%%%%%%%%%%%%%%%%%%%%%%%%%
\textbf{Значај теме и области:}

    \emph{Хомотопна теорија типова} (ХоТТ) (енгл. Homotopy Type Theory) или \emph{Унивалентнe основе} (УO) (енгл. Univalent Foundations) укључује разне области: логику, теорију скупова, теорију категорија, хомотопну теорију, функционално програмирање, теорију типова и теорију доказа. ХоТТ/УO пружа алтернативан начин заснивања математике.~\emph{Интуиционистичка теорија типова} или \emph{Мартин-Луф теорија типова} (МЛТТ) представља стуб за ХоТТ/УO, и као таква битна је за разумевање ХоТТ/УO. Поред тога, МЛТТ пружа формалан систем чије варијанте имплементирају многи интерактивни доказивачи. Важно je joш напоменути да, МЛТТ поставља многа филозофска питања о природи математике и природи математичких објеката, као и да пружа практичне примене у формалној верификацији софтверских система осетљивих на грешке. \\

\textbf{Специфични циљ рада:}

    Формализовати основе МЛТТ у типски зависном функционалном језику Агда. То подразумева прецизно дефинисање основних индуктивних типова: празан тип $\mathbf{0}$, јединични тип $\mathbf{1}$, Булов тип $\mathbf{2}$, типови копроизвода $A + B$, типови зависних парова $\sum_{(x:A)} B(x)$, типови производа $A \times B$, типови зависних функција $\prod_{(x:A)} B(x)$, типови идентитета $x =_{A} y$, тип природних бројева $\mathds{N}$, тип целих бројева $\mathds{Z}$, коначни типови $\mathsf{Fin}_{n}$, типови листи $\mathsf{List}_{A}$, и др. За сваки прецизно дефинисани индуктивни тип, истражити и конструисати елементе за типове који су могу формирати од њих. Због Кари--Хауардове коресподенције, овај поступак ће одговарати записивању и конструктивистичком доказивању теорема. На тај начин изградити библиотеку, која ће омогућити даље изучавање и надоградњу за ХоТТ.\\

\end{minipage}
}
\end{flushleft}
\vspace{1cm}
%%%%%%%%%%%%%%%%%%%%%%%%%%%%%%%%%%%%%%%%%%%%%%%%%%%%%%%%%%%%%%%%%%%%%%%%%%%%%%%
% u donji red uneti:       ime i prezime, broj indeksa i modul studenta       %
%%%%%%%%%%%%%%%%%%%%%%%%%%%%%%%%%%%%%%%%%%%%%%%%%%%%%%%%%%%%%%%%%%%%%%%%%%%%%%%
\makebox[10cm][c]{\textbf{Андрија Урошевић, 1069/2022, информатика}}
%%%%%%%%%%%%%%%%%%%%%%%%%%%%%%%%%%%%%%%%%%%%%%%%%%%%%%%%%%%%%%%%%%%%%%%%%%%%%%%
% u donji red uneti:                   ime i prezime mentora				  %
%%%%%%%%%%%%%%%%%%%%%%%%%%%%%%%%%%%%%%%%%%%%%%%%%%%%%%%%%%%%%%%%%%%%%%%%%%%%%%%
Сагласан ментор \makebox[5cm][c]{\textbf{Сана Стојановић Ђурђевић}} \\
\rule[4mm]{10cm}{.05mm} \hfill\raisebox{4mm}{\makebox[5cm][l]{.\dotfill.}} \\
\raisebox{1cm}%
[9mm][0mm]{\makebox[10cm][c]{\textit{(име и презиме студента, бр. индекса, модул)}}} \\
\makebox[10cm]{}\\
\vspace{-1cm}\\
\rule[2cm]{6.5cm}{.05mm} \hfill\rule[2cm]{6.5cm}{.05mm}\\
\vspace{-2.4cm}\\
\raisebox{2cm}{\makebox[6.5cm][c]{\textit{(својеручни потпис студента)}}}
\hfill\raisebox{2cm}{\makebox[6.5cm][c]{\textit{(својеручни потпис ментора)}}}\\
\vspace{-2cm}\\
%%%%%%%%%%%%%%%%%%%%%%%%%%%%%%%%%%%%%%%%%%%%%%%%%%%%%%%%%%%%%%%%%%%%%%%%%%%%%%%
% u donji red uneti datum podnosenja molbe									  %
%%%%%%%%%%%%%%%%%%%%%%%%%%%%%%%%%%%%%%%%%%%%%%%%%%%%%%%%%%%%%%%%%%%%%%%%%%%%%%%
\makebox[5.5cm][c]{\textbf{<датум>}}\makebox[5.5cm]{}  Чланови комисије\\
%%%%%%%%%%%%%%%%%%%%%%%%%%%%%%%%%%%%%%%%%%%%%%%%%%%%%%%%%%%%%%%%%%%%%%%%%%%%%%%
% POPUNJAVA MENTOR (rucno ili na sledeci nacin):							  %
% u donji red umesto .\dotfill. upisati podatke o 1. clanu komisije		      %
%%%%%%%%%%%%%%%%%%%%%%%%%%%%%%%%%%%%%%%%%%%%%%%%%%%%%%%%%%%%%%%%%%%%%%%%%%%%%%%
\rule[4mm]{5.5cm}{.05mm}\makebox[5.5cm]{} 1.\makebox[6cm][l]{.\dotfill.}\\
\vspace{-8mm}\\
\raisebox{4mm}%														
[7mm][0mm]{\makebox[5.5cm][c]{\textit{(датум подношења молбе)}}}\makebox[5.5cm]{}
%%%%%%%%%%%%%%%%%%%%%%%%%%%%%%%%%%%%%%%%%%%%%%%%%%%%%%%%%%%%%%%%%%%%%%%%%%%%%%%
% POPUNJAVA MENTOR (rucno ili na sledeci nacin): 							  %
% u donji red umesto .\dotfill. upisati podatke o 2. clanu komisije           %
%%%%%%%%%%%%%%%%%%%%%%%%%%%%%%%%%%%%%%%%%%%%%%%%%%%%%%%%%%%%%%%%%%%%%%%%%%%%%%%
2.\makebox[6cm][l]{.\dotfill.}\\

\vspace{1cm}


\begin{flushleft}
%%%%%%%%%%%%%%%%%%%%%%%%%%%%%%%%%%%%%%%%%%%%%%%%%%%%%%%%%%%%%%%%%%%%%%%%%%%%%%%
% u donji red upisati              katedru									  %
%%%%%%%%%%%%%%%%%%%%%%%%%%%%%%%%%%%%%%%%%%%%%%%%%%%%%%%%%%%%%%%%%%%%%%%%%%%%%%%
Катедра \makebox[9.5cm][l]{\textbf{за рачунарство и информатику}} је сагласна са предложеном темом.
\vspace{-3mm}
\hspace*{13mm} \rule[2.3cm]{9.5cm}{.05mm}\\
\vspace{-1cm}
%%%%%%%%%%%%%%%%%%%%%%%%%%%%%%%%%%%%%%%%%%%%%%%%%%%%%%%%%%%%%%%%%%%%%%%%%%%%%%
% POPUNJAVA SEF KATEDRE                                                      %
%%%%%%%%%%%%%%%%%%%%%%%%%%%%%%%%%%%%%%%%%%%%%%%%%%%%%%%%%%%%%%%%%%%%%%%%%%%%%%
\makebox[6.5cm][c]{} \hfill\makebox[6.5cm][c]{}\\
\rule[4mm]{6.5cm}{.05mm} \hfill\rule[4mm]{6.5cm}{.05mm}\\
\vspace{-5mm}
\makebox[6.5cm][c]{\textit{(шеф катедре)}} \hfill\makebox[6.5cm][c]{\textit{(датум одобравања молбе)}}
\end{flushleft}
\end{document} 
