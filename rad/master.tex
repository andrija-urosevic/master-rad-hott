\documentclass[12pt,oneside]{memoir}

\usepackage[biblatex]{matfmaster}

\usepackage{cmsrb}

\bib{master}

\autor{Андрија Д. Урошевић}
\naslov{Хомотопна теорија типова}
\godina{2024}

\mentor{др Сана \textsc{Стојановић-Ђурђевић}, доцент\\ Универзитет у Београду, Математички факултет}
\komisijaA{др Филип \textsc{Марић}, редовни професор\\ Универзитет у Београду, Математички факултет}
\komisijaB{др Лаза \textsc{Лазић}, доцент\\ Универзитет у Београду, Математички факултет}

\datumodbrane{29. фебруар 2024.}

\apstr{%
    Напиши апстракт на крају
}

\kljucnereci{хомотопна теорија типова, интерактивно доказивање, агда}

\begin{document}
% ==============================================================================
\frontmatter\
% ==============================================================================

\naslovna\

\komisija\

\posveta{Мами, тати и деди}

\apstrakt\

\tableofcontents*

% ==============================================================================
\mainmatter\
% ==============================================================================

% ------------------------------------------------------------------------------
\chapter{Увод}
% ------------------------------------------------------------------------------

\begin{itemize}
    \item{Хомотопна теорија типова = интуиционистичка теорија типова + високи индуктивни типови + аксиома унивалентности.}
    \item{Пер Мартин-Луф теорија типова се заснива на интиуционистичком програму који је настао по Брауверу.}
    \item{Математичко резтоновање је људска активност и математика је језик у коме се математичке идеје преносе.}
    \item{Фундаментална људска активност.}
    \item{Конструктивна теорија је \textit{доказно релевантна}, тј. доказ је математички објекат као и сваки други.}
    \item{Тврђења можемо интерпретирати као типове, те ће доказ представљати \textit{проверу типа}, тј. конструисање терма одређеног типа. (Јако битна уврнута идеја)}
    \item{Запажање: Хомотопна тероја и теорија типова представљају исту ствар.}
    \item{Хомотопна теорија се бави непрекидним пресликавањима која су \textit{хомотопна} између себе, тј. могу се ``непрекидно деформисати'' једна у друге.}
    \item{Тројство израчуњивости: Програмерска интерпретација, хомотопна интерпретација и логичка интерпретација.}
    \item{Типско расуђивање $t : T$ читамо као $t$ је терм типа $Т$ или терм $t$ настањује $T$. У програмерској интерпретацији тип представља тип, док терм неког типа представља израз тог типа. У хомотопној интерпретацији тип представља простор, док терм неког типа представља тачку у том простору.}
    \item{Пример јединичног типа $\mathbb{1}$: јединични (\texttt{unit} у програмерском смислу), јединствени ($The$ у логичком смислу), и контрактибилни (у хомотопном смислу) тип.}
    \item{Интенционални и eкстенционални типови? (нешто чуно, проучити)}
    \item{Раселов парадокс као мотивација за теорију типова.}
\end{itemize}

\section{Интуиционистичка теорија типова}

Интуиционистичка теорија типова или Пер Мартин-Луф теорија типова је математичка теорија конструкција. Тип представља врсту конструкције. Елемент, терм или тачка представља резултат конструкције неког типа. Прецизније, елемент $a$ типа $A$ записујемо као $a : A$, и кажемо да елемент $a$ настањује тип $A$. Битно је напоменути да терм не може да ``живи самостално'' тј. терм увек мора да настањује неки тип. 

Конструкција типова се састоји из низа дедуктивних \emph{правила закључивања}. Правило закључивања записујемо као
\begin{prooftree}
    \AxiomC{$\mathcal{H}_1$}
    \AxiomC{$\mathcal{H}_2$}
    \AxiomC{$\ldots$}
    \AxiomC{$\mathcal{H}_n$}
    \QuaternaryInfC{$\mathcal{C}$}
\end{prooftree}
где расуђивања $\mathcal{H}_1$, $\mathcal{H}_2, \ldots, \mathcal{H}_n$ називамо \emph{премисе} или \emph{хипотезе}, а расуђивање $\mathcal{C}$ називамо \emph{закључак}.

\begin{definition}
    Свако \emph{расуђивање} је облика $\Gamma\vdash \mathcal{J}$, где је $\Gamma$ \emph{контекст} и $\mathcal{J}$ \emph{теза} расуђивања. Теза може имати четири врсте расуђивања и то су:
    \begin{enumerate}[(i)]
        \item{$A$ је \emph{(добро-формиран) тип} у контексту $\Gamma$. \[\Gamma\vdash A~\type\]}
        \item{$A$ и $B$ су \emph{расуђивачки једнаки типови} у контексту $\Gamma$. \[\Gamma\vdash A~\dot{=}~B~\type\]}
        \item{$a$ је \emph{елемент} типа $A$ у контексту $\Gamma$. \[\Gamma\vdash a : A\]}
        \item{$a$ и $b$ су \emph{расуђивачки једнаки елементи} типа $A$ у контексту $\Gamma$. \[\Gamma\vdash a~\dot{=}_A~b : A\]}
    користећи правила закључивања теорије типова.
    \end{enumerate}
Контекст је коначна листа \emph{декларисаних променљивих} облика \[x_1 : A_1, x_2 : A_2 (x_1), \ldots, x_n : A_n(x_1, \ldots, x_{n-1}),\] под условом да за свако $1 \le k \le n$ можемо да изведемо расуђивање \[x_1 : A_1, x_2 : A_2(x_1), \ldots, x_{k-1} : A_{k-1}(x_1, \ldots, x_{k-2}) \vdash A_k(x_1, x_2, \ldots, x_{k-1}),\] применом правила закључивања.
\end{definition}

Из дефиниције контекста можемо видети да неки типови зависе од других термова. На пример, $A_2(x_1)$ зависи од $x_1 : A_1$, тј. за разне термове $x_1 : A_1$ имамо разне типове $A_2(x_1)$. Ову идеју можемо уопштити помоћу следеће две дефиниције:

\begin{definition}
    Нека је тип $A$ у контексту $\Gamma$. \emph{Фамилија} типова над $A$ у контексту $\Gamma$ је тип $B(x)$ у контексту $\Gamma, x : A$, тј.
    \[\Gamma, x : A \vdash B(x)~\type.\]
    Кажемо да је $B$ фамилија типова над $A$ у контексту $\Gamma$. Алтернативно, кажемо да је $B(x)$ тип индексиран са $x : A$ у контексту $\Gamma$.
\end{definition}

\begin{definition}
    Нека је $B$ фамилија типова над $A$ у контексут $\Gamma$. \emph{Секција} фамилије $B$ над типом $A$ у контексту $\Gamma$ је елемент типа $B(x)$ у контексту $\Gamma, x : A$, тј.
    \[\Gamma, x : A \vdash b(x) : B(x).\]
    Кажемо да је $b$ секција фамилије $B$ над $A$ у контексту $\Gamma$. Алтернативно, кажемода да је $b(x)$ елемент типа $B(x)$ индексиран са $x : A$ у контексту $\Gamma, x : A$. 
\end{definition}

\subsection{Правила закључивања}

Нека правила закључивања:

\begin{samepage}
    \begin{center}
        \begin{minipage}{.2\textwidth}
            \begin{prooftree}
                \AxiomC{$\Gamma\vdash A~\textrm{type}$}
                \UnaryInfC{$\Gamma\vdash A \dot{=} A~\textrm{type}$}
            \end{prooftree}
        \end{minipage}
        \begin{minipage}{.25\textwidth}
            \begin{prooftree}
                \AxiomC{$\Gamma\vdash A\dot{=} A'~\textrm{type}$}
                \UnaryInfC{$\Gamma\vdash A'\dot{=} A~\textrm{type}$}
            \end{prooftree}
        \end{minipage}
        \begin{minipage}{.5\textwidth}
            \begin{prooftree}
                \AxiomC{$\Gamma\vdash A\dot{=} A'~\textrm{type}$}
                \AxiomC{$\Gamma\vdash A'\dot{=} A''~\textrm{type}$}
                \BinaryInfC{$\Gamma\vdash A\dot{=} A''~\textrm{type}$}
            \end{prooftree}
        \end{minipage}
        \\*
        \bigskip\
        \begin{minipage}{.2\textwidth}
            \begin{prooftree}
                \AxiomC{$\Gamma\vdash a:A$}
                \UnaryInfC{$\Gamma\vdash a\dot{=} a : A$}
            \end{prooftree}
        \end{minipage}
        \begin{minipage}{.25\textwidth}
            \begin{prooftree}
                \AxiomC{$\Gamma\vdash a\dot{=} a':A$}
                \UnaryInfC{$\Gamma\vdash a'\dot{=} a: A$}
            \end{prooftree}
        \end{minipage}
        \begin{minipage}{.5\textwidth}
            \begin{prooftree}
                \AxiomC{$\Gamma\vdash a\dot{=} a' : A$}
                \AxiomC{$\Gamma\vdash a'\dot{=} a'': A$}
                \BinaryInfC{$\Gamma\vdash a\dot{=} a'': A$}
            \end{prooftree}
        \end{minipage}
        %\end{small}
    \end{center}
\end{samepage}



\subsection{Зависни типови}

\section{Аксиома унивалентности}

% ------------------------------------------------------------------------------
\chapter{Разрада}
\label{chp:razrada}
% ------------------------------------------------------------------------------

% ------------------------------------------------------------------------------
\chapter{Закључак}
% ------------------------------------------------------------------------------

% ------------------------------------------------------------------------------
% Literatura
% ------------------------------------------------------------------------------
\nocite{*}
\literatura\

% ==============================================================================
\backmatter\
% ==============================================================================

% ------------------------------------------------------------------------------
% Biografija kandidata
\begin{biografija}
\textbf{Вук Стефановић Караџић} (\emph{Тршић, 26. октобар/6. новембар
  1787. — Беч, 7. фебруар 1864.}) био је српски филолог, реформатор
српског језика, сакупљач народних умотворина и писац првог речника
српског језика.  Вук је најзначајнија личност српске књижевности прве
половине XIX века. Стекао је и неколико почасних доктората.
Учествовао је у Првом српском устанку као писар и чиновник у
Неготинској крајини, а након слома устанка преселио се у Беч,
1813. године. Ту је упознао Јернеја Копитара, цензора словенских
књига, на чији је подстицај кренуо у прикупљање српских народних
песама, реформу ћирилице и борбу за увођење народног језика у српску
књижевност. Вуковим реформама у српски језик је уведен фонетски
правопис, а српски језик је потиснуо славеносрпски језик који је у то
време био језик образованих људи. Тако се као најважније године Вукове
реформе истичу 1818., 1836., 1839., 1847. и 1852.
\end{biografija}
% ------------------------------------------------------------------------------

\end{document} 
