\textsc{Agda} програми се развијају интерактивно, што значи да је могуће верификовати ограничења типовa (енгл. \emph{type check}) иако код није комплетно написан. Програм који извршава верификацију ограничења типова назива се \emph{проверивач типова} (енгл.~\emph{typechecker}). Код може садржати празнине које се могу допунити касније. Многи текстуални едитори подржавају интерактивни развој \textsc{Agda} програма: \textsc{Emacs}, \textsc{Vim}, \textsc{Atom}, \textsc{Visual Studio Code}.

Када се интерактивно развија \textsc{Agda} програм користе се команде које пружају информације од проверивача типова, као и начине на које можемо обрађивати празнине. Неке од корисних команда су:

\begin{itemize}
    \item[\texttt{C-c C-l}:]{Учитава фајл и извршава верификацију ограничења типова. Свако појављивање знак \texttt{?} замењује са новом празнином.}
    \item[\texttt{C-c C-d}:]{Одређује тип датог израза.}
    \item[\texttt{C-c C-n}:]{Нормализује дати израз.}
    \item[\texttt{C-c C-f}:]{Поставља курсор на прву следећу празнину.}
    \item[\texttt{C-c C-b}:]{Поставља курсор на прву претходну празнину.}
    \item[\texttt{C-c C-,}:]{Приказује очекивани тип празнине на којој се налази курсор, као и типове свих променљивих из контекста.}
    \item[\texttt{C-c C-c}:]{Раздваја променљиву на све могуће случајеве.}
    \item[\texttt{C-c C-SPC}:]{Замењује празнину датим изразом, ако је дати израз одговарајућег типа.}
    \item[\texttt{C-c C-r}:]{Прерађује празнину тако што је замени са изразом заједно са новим одговарајућим празнинама.}
    \item[\texttt{C-c C-a}:]{Аутоматски покушава да пронађе одговарајући израз за празнину на којој се налази курсор.}
    \item[\texttt{C-x C-c}:]{Преводи програм.}
\end{itemize}

Прођимо кроз један основни пример. Како се име сваког \textsc{Agda} фајла завршава екстензијом \texttt{.agda}, нека се наш фајл зове \texttt{Tutorial.agda}.

На почетку фајла описујемо неке мета-податке:
\begin{code}%
\>[0]\AgdaSymbol{\{-\#}\AgdaSpace{}%
\AgdaKeyword{OPTIONS}\AgdaSpace{}%
\AgdaPragma{--without-K}\AgdaSpace{}%
\AgdaPragma{--safe}\AgdaSpace{}%
\AgdaSymbol{\#-\}}\<%
\end{code}
У овом случају то су опције \texttt{--without-K} и \texttt{--safe}, које обезбеђују сигурну аксиоматизацију унивалентности. Без ових опција дошло би до контрадикције.

\textsc{Agda} програм је сачињен од \emph{модула}. Сваки модул има своје име (које мора да се поклопи са именом фајла) као и низ декларација. Модул за наш фајл \texttt{Tutorial.agda} започињемо на следећи начин:
\begin{code}%
\>[0]\AgdaKeyword{module}\AgdaSpace{}%
\AgdaModule{Tutorial}\AgdaSpace{}%
\AgdaKeyword{where}\<%
\end{code}

Једна врста декларација је увожење других модула. Тако, на пример, можемо увести модул \texttt{Universes} на следећи начин: 
\begin{code}%
\>[0]\AgdaKeyword{open}\AgdaSpace{}%
\AgdaKeyword{import}\AgdaSpace{}%
\AgdaModule{Universes}\AgdaSpace{}%
\AgdaKeyword{public}\<%
\end{code}

Друга врста декларација је декларисање променљивих. На пример, можемо рећи да $\mathcal{U}$, $\mathcal{V}$ и $\mathcal{W}$ представљају универзум:
\begin{code}%
\>[0]\AgdaKeyword{variable}\AgdaSpace{}%
\AgdaGeneralizable{𝓤}\AgdaSpace{}%
\AgdaGeneralizable{𝓥}\AgdaSpace{}%
\AgdaGeneralizable{𝓦}\AgdaSpace{}%
\AgdaSymbol{:}\AgdaSpace{}%
\AgdaPostulate{Universe}\<%
\end{code}

Једна од кључних декларација је декларација индуктивних типова. Декларисање индуктивних типова подразумева навођење имена тог типа, начин на који се он формира, као и начин за конструисање каноничних елемената тог типа. У теорији типова, то смо називали правило формирања типова и правила конструисања. Подсетимо се правила формирања и правила конструисања типа природних бројева $\N$. Тип природних бројева $\N$ може да се формира из празног контекста, односно $\N : \mathcal{U}_0$, као и да су му једини конструктори $\zeroN : \N$ и $\succN : \N \to \N$. У језику \textsc{Agda} то записујемо као: 
\begin{code}%
\>[0]\AgdaKeyword{data}\AgdaSpace{}%
\AgdaDatatype{ℕ}\AgdaSpace{}%
\AgdaSymbol{:}\AgdaSpace{}%
\AgdaPrimitive{𝓤₀}\AgdaSpace{}%
\AgdaOperator{\AgdaFunction{̇}}\AgdaSpace{}%
\AgdaKeyword{where}\<%
\\
\>[0][@{}l@{\AgdaIndent{0}}]%
\>[4]\AgdaInductiveConstructor{zero}\AgdaSpace{}%
\AgdaSymbol{:}\AgdaSpace{}%
\AgdaDatatype{ℕ}\<%
\\
%
\>[4]\AgdaInductiveConstructor{succ}\AgdaSpace{}%
\AgdaSymbol{:}\AgdaSpace{}%
\AgdaDatatype{ℕ}\AgdaSpace{}%
\AgdaSymbol{→}\AgdaSpace{}%
\AgdaDatatype{ℕ}\<%
\\
\>[0]\<%
\\
\>[0]\AgdaSymbol{\{-\#}\AgdaSpace{}%
\AgdaKeyword{BUILTIN}\AgdaSpace{}%
\AgdaKeyword{NATURAL}\AgdaSpace{}%
\AgdaDatatype{ℕ}\AgdaSpace{}%
\AgdaSymbol{\#-\}}\<%
\end{code}

Поред правила формирања и правила конструисања, за комплетну спецификацију индуктивних типова имали смо и правило индукције и правило израчунавања. Правило индукције типа природних бројева $\N$, у језику \textsc{Agda} записујемо као:
\begin{code} %
\>[0]\AgdaFunction{ℕ-induction}%
\>[29I]\AgdaSymbol{:}\AgdaSpace{}%
\AgdaSymbol{(}\AgdaBound{P}\AgdaSpace{}%
\AgdaSymbol{:}\AgdaSpace{}%
\AgdaDatatype{ℕ}\AgdaSpace{}%
\AgdaSymbol{→}\AgdaSpace{}%
\AgdaPrimitive{𝓤₀}\AgdaSpace{}%
\AgdaOperator{\AgdaFunction{̇}}\AgdaSpace{}%
\AgdaSymbol{)}\<%
\\
\>[.][@{}l@{}]\<[29I]%
\>[12]\AgdaSymbol{→}\AgdaSpace{}%
\AgdaBound{P}\AgdaSpace{}%
\AgdaNumber{0}\<%
\\
%
\>[12]\AgdaSymbol{→}\AgdaSpace{}%
\AgdaSymbol{((}\AgdaBound{n}\AgdaSpace{}%
\AgdaSymbol{:}\AgdaSpace{}%
\AgdaDatatype{ℕ}\AgdaSymbol{)}\AgdaSpace{}%
\AgdaSymbol{→}\AgdaSpace{}%
\AgdaBound{P}\AgdaSpace{}%
\AgdaBound{n}\AgdaSpace{}%
\AgdaSymbol{→}\AgdaSpace{}%
\AgdaBound{P}\AgdaSpace{}%
\AgdaSymbol{(}\AgdaInductiveConstructor{succ}\AgdaSpace{}%
\AgdaBound{n}\AgdaSymbol{))}\<%
\\
%
\>[12]\AgdaSymbol{→}\AgdaSpace{}%
\AgdaSymbol{(}\AgdaBound{n}\AgdaSpace{}%
\AgdaSymbol{:}\AgdaSpace{}%
\AgdaDatatype{ℕ}\AgdaSymbol{)}\AgdaSpace{}%
\AgdaSymbol{→}\AgdaSpace{}%
\AgdaBound{P}\AgdaSpace{}%
\AgdaBound{n}\<%
\\
\>[0]\<%
\end{code}
Са друге стране, правила израчунавања записујемо као:
\begin{code}%
\>[0]\AgdaFunction{ℕ-induction}\AgdaSpace{}%
\AgdaBound{P}\AgdaSpace{}%
\AgdaBound{p₀}\AgdaSpace{}%
\AgdaBound{pₛ}\AgdaSpace{}%
\AgdaInductiveConstructor{zero}%
\>[29]\AgdaSymbol{=}\AgdaSpace{}%
\AgdaBound{p₀}\<%
\\
\>[0]\AgdaFunction{ℕ-induction}\AgdaSpace{}%
\AgdaBound{P}\AgdaSpace{}%
\AgdaBound{p₀}\AgdaSpace{}%
\AgdaBound{pₛ}\AgdaSpace{}%
\AgdaSymbol{(}\AgdaInductiveConstructor{succ}\AgdaSpace{}%
\AgdaBound{n}\AgdaSymbol{)}\AgdaSpace{}%
\AgdaSymbol{=}\AgdaSpace{}%
\AgdaBound{pₛ}\AgdaSpace{}%
\AgdaBound{n}\AgdaSpace{}%
\AgdaSymbol{(}\AgdaFunction{ℕ-induction}\AgdaSpace{}%
\AgdaBound{P}\AgdaSpace{}%
\AgdaBound{p₀}\AgdaSpace{}%
\AgdaBound{pₛ}\AgdaSpace{}%
\AgdaBound{n}\AgdaSymbol{)}\<%
\end{code}

У многим случајевима \textsc{Agda} може помоћи при декларисању правила израчунавања. Како она описују начин конструисања елемента који оправдава правило индукције можемо поћи од:
\begin{code}%
\>[0]\AgdaFunction{ℕ-induction}\AgdaSpace{}%
\AgdaBound{P}\AgdaSpace{}%
\AgdaBound{p₀}\AgdaSpace{}%
\AgdaBound{pₛ}\AgdaSpace{}%
\AgdaBound{n}\AgdaSpace{}%
\AgdaSymbol{=}\AgdaSpace{}%
\AgdaHole{\{!\ \ \ !\}}\<%
\end{code}
Командом \texttt{C-c C-c} раздвајамо \texttt{n} на случајеве:
\begin{code}%
\>[0]\AgdaFunction{ℕ-induction}\AgdaSpace{}%
\AgdaBound{P}\AgdaSpace{}%
\AgdaBound{p₀}\AgdaSpace{}%
\AgdaBound{pₛ}\AgdaSpace{}%
\AgdaInductiveConstructor{zero}%
\>[29]\AgdaSymbol{=}\AgdaSpace{}%
\AgdaHole{\{!\ \ \ !\}}\<%
\\
\>[0]\AgdaFunction{ℕ-induction}\AgdaSpace{}%
\AgdaBound{P}\AgdaSpace{}%
\AgdaBound{p₀}\AgdaSpace{}%
\AgdaBound{pₛ}\AgdaSpace{}%
\AgdaSymbol{(}\AgdaInductiveConstructor{succ}\AgdaSpace{}%
\AgdaBound{n}\AgdaSymbol{)}\AgdaSpace{}%
\AgdaSymbol{=}\AgdaSpace{}%
\AgdaHole{\{!\ \ \ !\}}\<%
\end{code}
Командом \texttt{C-c C-а} аутоматски решавамо први случај, док на други примењујемо \texttt{C-c C-r} над изразом $p_s$:
\begin{code}%
\>[0]\AgdaFunction{ℕ-induction}\AgdaSpace{}%
\AgdaBound{P}\AgdaSpace{}%
\AgdaBound{p₀}\AgdaSpace{}%
\AgdaBound{pₛ}\AgdaSpace{}%
\AgdaInductiveConstructor{zero}%
\>[29]\AgdaSymbol{=}\AgdaSpace{}%
\AgdaBound{p₀}\<%
\\
\>[0]\AgdaFunction{ℕ-induction}\AgdaSpace{}%
\AgdaBound{P}\AgdaSpace{}%
\AgdaBound{p₀}\AgdaSpace{}%
\AgdaBound{pₛ}\AgdaSpace{}%
\AgdaSymbol{(}\AgdaInductiveConstructor{succ}\AgdaSpace{}%
\AgdaBound{n}\AgdaSymbol{)}\AgdaSpace{}%
\AgdaSymbol{=}\AgdaSpace{}%
\AgdaBound{pₛ}\AgdaSpace{}%
\AgdaHole{\{!\ \ \ !\}}\AgdaSpace{}%
\AgdaHole{\{!\ \ \ !\}}\<%
\end{code}
На првом пољу, командом \texttt{C-c C-,}, сазнајемо да је потребно конструисати тип $\N$, као и да је у контексту $n : \N$. Због тога, једноставно, попуњавамо поље са $n$ командом \texttt{C-c C-SPC}. На другом пољу, командом \texttt{C-c C-,}, сазнајемо да је потребно конструисати тип $P(n)$ за $n : \N$. Приметимо да је \AgdaFunction{ℕ-induction} функција која враћа $P(n)$, те примењујемо команду \texttt{C-c C-r} над \AgdaFunction{ℕ-induction} $P$.
\begin{code}%
\>[0]\AgdaFunction{ℕ-induction}\AgdaSpace{}%
\AgdaBound{P}\AgdaSpace{}%
\AgdaBound{p₀}\AgdaSpace{}%
\AgdaBound{pₛ}\AgdaSpace{}%
\AgdaInductiveConstructor{zero}%
\>[29]\AgdaSymbol{=}\AgdaSpace{}%
\AgdaBound{p₀}\<%
\\
\>[0]\AgdaFunction{ℕ-induction}\AgdaSpace{}%
\AgdaBound{P}\AgdaSpace{}%
\AgdaBound{p₀}\AgdaSpace{}%
\AgdaBound{pₛ}\AgdaSpace{}%
\AgdaSymbol{(}\AgdaInductiveConstructor{succ}\AgdaSpace{}%
\AgdaBound{n}\AgdaSymbol{)}\AgdaSpace{}%
\AgdaSymbol{=}\AgdaSpace{}%
\AgdaBound{pₛ}\AgdaSpace{}%
\AgdaBound{n}\AgdaSpace{}%
\AgdaSymbol{(}\AgdaFunction{ℕ-induction}\AgdaSpace{}%
\AgdaBound{P}\AgdaSpace{}%
\AgdaHole{\{!\ \ \ !\}}\AgdaSpace{}%
\AgdaHole{\{!\ \ \ !\}}\AgdaSpace{}%
\AgdaHole{\{!\ \ \ !\}}\AgdaSymbol{)}\<%
\end{code}
Преостала поља можемо једноставно попунити аутоматски помоћу команде \texttt{C-c C-a}, или разматрањем траженог типа и контекста помоћу команде \texttt{C-c C-,}, а онда уписивањем одговарајућих израза помоћу команде \texttt{C-c C-SPC}.
