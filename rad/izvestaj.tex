\documentclass{letter}
\usepackage[OT2]{fontenc}
\usepackage[serbian]{babel}
\usepackage[a4paper, total={6in, 8in}]{geometry}

\pagenumbering{gobble}

\begin{document}

Универзитет у Београду\\
Математички факултет\\
Катедра за рачунарство и информатику

\vspace{2cm}

Предмет:\\
\textbf{Извештај о прегледу мастер рада~,,Формализациjа интуиционистичке теориjе типова као увод у хомотопну теориjу типова'' кандидата Андрије Урошевића}

\vspace{0.6cm}

На седници Наставно-научног већа Математичког факултета, одржаној 28.06.2023. године, именовани смо за чланове комисије за преглед и одбрану мастер рада~,,Формализациjа интуиционистичке теориjе типова као увод у хомотопну теориjу типова'' кандидатa Андрије Урошевића, студентa мастер студија Информатике на Математичком факултету.

Текст се састоји из пет глава и списка референци у укупном обиму од 61 стране. Прва, уводна глава описује филозофију и историју теорије типова, уводи мотивацију и циљеве мастер рада и даје преглед релевантне литературе. Друга глава пружа преглед кључних појмова интуиционистичке теорије типова. Трећа глава описује програмски језик \textsc{Agda}. Четврта глава описује садржај и имплементацију библиотеке \textsc{InTT} (\textsc{In}tucionistic \textsc{T}ype \textsc{T}heory) у којој су дефинисани основни појмови интуиционистичке теорије типова. У петој глави изводе се закључци.

\vspace{0.4cm}

\textbf{Закључак}\\
Увидом у текст~,,Формализациjа интуиционистичке теориjе типова као увод у хомотопну теориjу типова'' Андрије Урошевића дошли смо до закључка да рад у потпуности испуњава захтеве који се постављају при изради мастер рада и предлажемо Катедри да одобри јавну одбрану рада.

\vspace{1.2cm}

У Београду

30. август 2024.

\vspace{0.5cm}

Комисија:

др Сана Стојановић Ђурђевић, доцент

проф. др Филип Марић, редовни професор

др Иван Чукић, доцент

\end{document}

